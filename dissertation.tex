\documentclass{article}
\usepackage[utf8]{inputenc}
\usepackage{hyperref}
\usepackage{graphicx}
\usepackage{todonotes}
\hypersetup{hidelinks}
\usepackage[style=authoryear,sorting=nyt]{biblatex}
\bibliography{{dissertation}}
\addbibresource{dissertation.bib}

\title {Variations of economic response in the COVID-19 crisis}
\date {July 2020}
\author {Mihai Chereji
    %\thanks{For my life partner and greatest cheerleader, Cristina, the impressive faculty at IIPP and everyone that's made this possible}
}
\begin{document}
% \tableofcontents
\maketitle
\section{Introduction}
\label{Para:Intro}
    The COVID-19 pandemic that has erupted in 2019 and is still infecting and killing thousands at the time of writing has quickly morphed from a public health emergency to also be an economic crisis of a magnitude the world has not encountered in decades. The crisis was initially triggered by the lockdown measures enacted as a way to contain the virus. But even as economies reopen, economic forecasts are grim (IMF's latest World Economic Outlook projects a global GDP contraction of 4.9\% for 2020 \footcite[]{noauthor_world_2020}), and models are being built to take into consideration \emph{scarring of beliefs}\autocite{kozlowski_scarring_2020}. All relevant economic indicators \todo[]{add indicators here - Unemployment, Fall in GDP, Fall in worked hours, emphasize consumer spending} have depreciated across OECD countries a lot faster and a lot more than during the previous large shock to the economy, the Great Financial Crisis (GFC) of 2008-2009, and the subsequent Eurozone Sovereign Debt Crisis that followed it.

\label{Para:Consequences}
    Tasked with such a gargantuan challenge, governments' role in the recovery from this crisis (dubbed by the IMF The Great Lockdown \autocite{gopinath_great_2020}, a term which I shall use as well) has become readily apparent even to the most ardent of state non-interventionists. The shape and speed of economic recoveries depend on the policies that governments have enacted in the first few months of the crisis, and the fiscal stimulus packages or austerity measures aimed at taming the public debt incurred during the months when their economy was locked down that will follow.

% • Motivation: Describe briefly the big social challenge you will address & why it is of great importance (Tip: here you can cite academic sources and non-academic sources to demonstrate that this topic is important for a wider audience and not just an academic exercise)
% • Existing literature: Provide a very brief overview of the different views on the topic, i.e. causes, consequences, proposed solutions, country/case study focus
% • Your contribution: Highlight the main gap in the literature (i.e. form your research question), where you will focus to fill this gap (case study/country selection), and how you will try to answer your question (methodology)
% • Results & Conclusions: Provide a brief overview of your key findings and how these relate to existing studies.
% • Thesis structure outline: Provide a very brief outline of the next sections (e.g. “Section two provides a review of the relevant literature on...”)
% • Overall: The introduction is meant to be a condensed description of the whole thesis


\label{para:CPE}
    Comparative Political Economy has long tried to ascertain how structural and institutional differences amongst advanced capitalist economies lead to different policy choices and outcomes and how these might be correlated. Some research programmes in the field, such as the predominant literature known as Varieties of Capitalism (\textbf{VofC}), introduced by Hope and Soskice \autocite*{hall_introduction_2001} emphasizes the firm and the relationship between firms as the main differentiating factor amongst two highly stylised variations, called \textbf{Liberal Market Economies} (\textbf{LME}), in which the relationship between companies is one of market competition, and \textbf{Coordinated Market Economies} (\textbf{CME}), in which companies are strategically coordinated. It also acknowledges the existence of \textbf{Mixed Marked Economies} (\textbf{MME}), but stresses their institutional incoherence. The literature has later been enhanced with different other variations, amongst them  \textbf{Hierarchical Market Economies} (\textbf{HME}) \autocite{schneider_hierarchical_2009} and \textbf{Dependent Market Economies} (\textbf{DME}) \autocites{nolke_enlarging_2009}{ban_cocktail_2013}{soreg_patterns_2019}.


    \label{para:new}
    this one is just to test.

\section{Literature review}
As noted in \autocite{kruppe_labour_2014} \autocite{blanchard_hysteresis_1986}
% 2. Social Challenge & Literature Review
% Motivation:
% • Describe the economic/social/political issue you want to address
% • Why it is important?
% • Why it remains unsolved or overlooked?
% • Is it a global challenge or something that is specific to a certain region?
% Existing Literature:
% • What academic research says about this challenge? Theoretical & Empirical Insights
% • What case studies existing literature examines?
% • What methodologies they use? (Primary research: Interviews, questionnaires, etc.;
% Secondary research: Online source material, books, articles, reports, databases that
% contain statistical information)
% • What are they key findings and how different they are among different case studies?
% Open Questions:
% • Are there any gaps in the existing literature?
% • Overlooked aspects of the issue you discuss
% • Lack of research for specific regions
% • Recent developments that the literature has not yet addressed
\section{Research Methodology}
% 3. Research Design and Methodology
% • Building on the last part of the previous section, state which gap you will attempt to fill
% • Explain why this is important and potentially a priority over the rest
% • Explain why you choose to focus on (a) specific case studies/(y)
% • Is it a representative example or an outlier?
% • What should we expect to learn from this example?
% • Describe your research methodology
% • Are you going to follow a methodology similar to an existing study?
% • If you follow a comparative approach (i.e. use two or more case studies), what
% differences should we expect?
% • Describe data sources
% • If you use quantitative data as proxies to depict more abstract social aspects, it is
% important to clarify why each is a good (or the best available) proxy
\section{Main Findings}
% 4. Results and Discussion
% • Report results: Describe your main results
% • Discussion of your findings: If you analyse two or more case studies, following a
% comparative approach, did you observe the expected differences? If not, what is a
% possible explanation?
% • Comparison with existing literature: How your findings compare with existing
% arguments? Do they provide further support or they contradict existing studies?

\section{Conclusions}
% 5. Policy Implications
% • What do we learn from your findings?
% • Do they provide support to existing policy recommendations or they suggest that we
% need a new approach?
% • If the latter is true, what might be some new policy recommendations? (Tip: Be careful
% on whether your policy recommendations are generalisable or specific to your case
% studies. Refer to the previous section where you discuss case study selection)
% • Given the political environment in your case studies, how likely are such proposal to be
% acceptable?
% 6. Conclusions
% • Briefly restate the research question you addressed, what are the main shortcomings of existing studies, your approach, and main findings
% • Restate what do we learn from these results
% • Future work: Depending on whether your findings agree or contradict existing research,
% what are the next steps in this field?
\begin{figure}
\end{figure}
\listoftodos{TODOS}
\printbibliography{}
\end{document}