\documentclass{article}
\usepackage[utf8]{inputenc}
\usepackage{hyperref}
\usepackage{graphicx}
\usepackage{todonotes}
\usepackage[style=authoryear,sorting=nyt]{biblatex}
\addbibresource{dissertation.bib}

\title {Variations of economic response in the COVID-19 crisis}
\date {July 2020}
\author {Mihai Chereji
    \thanks{For my life partner and greatest cheerleader, Cristina, the impressive faculty at IIPP and everyone that's made this possible}
}
\begin{document}
% \tableofcontents
\maketitle
\section{Introduction}
    The COVID-19 pandemic that has erupted in 2019 and is still infecting and killing thousands at the time of writing has quickly morphed from a public health emergency to also be an economic crisis that the world has not encountered in decades, mainly caused by the lockdown measures enacted as a containment measure against it. All relevant economic indicators \todo[]{lookup and add indicators here - Unemployment, Fall in GDP, Fall in worked hours, consumer spending} have depreciated across a number of European countries and the United States a lot faster and more than the previous large shock to the economy, the Great Financial Crisis of 2008-2009 (and the subsequent Eurozone Sovereign Debt Crisis that followed it).




\section{Literature review}
\autocite{kruppe_labour_2014} \autocite{blanchard_hysteresis_1986}
\section{Research Methodology}
\section{Main Findings}
\section{Conclusions}
\begin{figure}
\end{figure}
\listoftodos{TODOS}
\printbibliography{}
\end{document}